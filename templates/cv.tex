%!TEX TS-program = xelatex
%!TEX encoding = UTF-8 Unicode
% Awesome CV LaTeX Template for CV/Resume
%
% This template has been downloaded from:
% https://github.com/posquit0/Awesome-CV
%
% Author:
% Claud D. Park <posquit0.bj@gmail.com>
% http://www.posquit0.com
%
% Template license:
% CC BY-SA 4.0 (https://creativecommons.org/licenses/by-sa/4.0/)
%


%------------------------------------------------------------------------------%
% CONFIGURATIONS                                                               %
%------------------------------------------------------------------------------%
% A4 paper size by default, use 'letterpaper' for US letter
\documentclass[10pt, letterpaper]{awesome-cv}

% Configure page margins with geometry
\geometry{left=1.4cm, top=.8cm, right=1.4cm, bottom=1.8cm, footskip=.5cm}

% Specify the location of the included fonts
\fontdir[fonts/]

% Color for highlights
% Awesome Colors: awesome-emerald, awesome-skyblue, awesome-red, awesome-pink, awesome-orange
%                 awesome-nephritis, awesome-concrete, awesome-darknight
\colorlet{awesome}{awesome-red}
% Uncomment if you would like to specify your own color
%\definecolor{awesome}{HTML}{002469}

% Colors for text
% Uncomment if you would like to specify your own color
% \definecolor{darktext}{HTML}{414141}
% \definecolor{text}{HTML}{333333}
%\definecolor{graytext}{HTML}{5e6062}
% \definecolor{lighttext}{HTML}{999999}

% Set false if you don't want to highlight section with awesome color
\setbool{acvSectionColorHighlight}{true}

% If you would like to change the social information separator from a pipe (|) to something else
\renewcommand{\acvHeaderSocialSep}{\quad\textbar\quad}


%------------------------------------------------------------------------------%
%	PERSONAL INFORMATION                                                         %
%	Comment any of the lines below if they are not required                      %
%------------------------------------------------------------------------------%
% Available options: circle|rectangle,edge/noedge,left/right
% \photo{./examples/profile.png}
\name{(((data.preamble.name.first)))}{(((data.preamble.name.last)))}
\position{((( data.preamble.current.role )))}
\address{(((data.preamble.address.street))), (((data.preamble.address.city))), (((data.preamble.address.country)))}

\email{(((data.preamble.contact.mail)))}
\homepage{(((data.preamble.contact.webpage)))}
\github{(((data.preamble.contact.github)))}
\gitlab{(((data.preamble.contact.gitlab)))}
\googlescholar{(((data.preamble.contact.googlescholar.id)))}{(((data.preamble.contact.googlescholar.name)))}
% \linkedin{linkedin-name}
% \stackoverflow{SO-id}{SO-name}
% \twitter{twitter-name}
% \skype{skype-id}
% \reddit{reddit-id}
% \extrainfo{extra informations}

%\quote{``Be the change that you want to see in the world."}


%-------------------------------------------------------------------------------
\begin{document}

% Print the header with above personal informations
% Give optional argument to change alignment(C: center, L: left, R: right)
\makecvheader

% Print the footer with 3 arguments(<left>, <center>, <right>)
% Leave any of these blank if they are not needed
\makecvfooter
  {\today}
  {(((data.preamble.name.first))) (((data.preamble.name.last)))~~~·~~~Curriculum Vitae}
  {\thepage}


%-------------------------------------------------------------------------------%
%-------------------------------------------------------------------------------%
%	                                CV Content                                    %
%-------------------------------------------------------------------------------%
%-------------------------------------------------------------------------------%

%-------------------------------------------------------------------------------%
%	                                Education                                     %
%-------------------------------------------------------------------------------%
((* set sect = data.sections | select_by_attr_name("title", "Education") *))
\cvsection{(((sect.title)))}
\begin{cventries}
((* for i in sect.entries *))
\cventry
  {(((i.degree | escape_tex )))}
  {(((i.school | escape_tex )))}
  {(((i.location | escape_tex )))}
  {(((i.dates)))}
  {((* if i.entries *))
    \begin{cvitems}
    ((* for j in i.entries -*))
    \item ((( j | escape_tex )))
    ((*- endfor *))
    \end{cvitems}
    ((* endif *))}
((* endfor *))
\end{cventries}

%-------------------------------------------------------------------------------%
%	                                Experience                                    %
%-------------------------------------------------------------------------------%
((* set sect = data.sections | select_by_attr_name("title", "Experience") *))
\cvsection{(((sect.title)))}
\begin{cventries}
((* for i in sect.entries *))
\cventry
  {(((i.institution | escape_tex)))}
  {(((i.title | escape_tex)))}
  {(((i.location | escape_tex)))}
  {(((i.dates | escape_tex)))}
  {(((i.description | escape_tex)))}
((* endfor *))
\end{cventries}

%-------------------------------------------------------------------------------%
%	                                Publications                                  %
%-------------------------------------------------------------------------------%
((* set  sect = data.sections | select_by_attr_name("title", "Papers") *))
\cvsection{((( sect.title )))}

A current publication list is also available from 
\href{https://orcid.org/((( data.preamble.contact.orcid )))}
     {Orcid (ID: ((( data.preamble.contact.orcid ))))}.
((* for i in sect.entries *))
\cvsubsection{((( i.title | escape_tex )))}
\begin{cvhonors}
  ((* for j in i.entries *))
  \cvhonor
    {((( j.title | escape_tex )))}
    {((( j.authors | author_filter(True) ))),
     {\slshape\color{awesome} ((( j.journal | escape_tex )))}((* if j.doi or j.bibcode or j.url -*)), ((*-  endif *))
     ((* if j.doi -*))
     {\slshape\color{graytext} doi: ((( j.doi | doi_to_url(j.doi, j.bibcode, 'tex') )))}
     ((*- elif j.bibcode -*))
     {\slshape\color{graytext} bibcode: ((( j.bibcode | doi_to_url(j.doi, j.bibcode, 'tex') )))}
     ((*- elif j.url -*))
     {\slshape\color{graytext} url: \href{((( j.url.link )))}{((( j.url.name )))}}((*- endif *))}
    {}
    {((( j.year )))}
  ((* endfor *))
\end{cvhonors}
((* endfor *))

%-------------------------------------------------------------------------------%
%	                                Fellowships                                   %
%-------------------------------------------------------------------------------%
((* set sect = data.sections | select_by_attr_name("title", "Research Fellowships Awarded") *))
\cvsection{((( sect.title | escape_tex )))}
\begin{cventries}
((* for i in sect.entries *))
\cventry
  {((( i.institution | escape_tex )))}
  {((( i.title | escape_tex )))}
  {((( i.location )))}
  {((( i.dates )))}
  {((( i.description | escape_tex )))}
((* endfor *))
\end{cventries}

%-------------------------------------------------------------------------------%
%	                                Service                                       %
%-------------------------------------------------------------------------------%
((* set sect = data.sections | select_by_attr_name("title", "Professional Service") *))
\cvsection{((( sect.title | escape_tex )))}
\begin{cventries}
((* for i in sect.entries *))
\cventry
  {((( i.role )))}
  {((( i.event )))}
  {}
  {((( i.dates )))}
  {((( i.description )))}
((* endfor *))
\end{cventries}

%-------------------------------------------------------------------------------%
%	                                Presentations                                 %
%-------------------------------------------------------------------------------%
((* set sect = data.sections | select_by_attr_name("title", "Presentations") *))
\cvsection{((( sect.title | escape_tex )))}
((* for i in sect.entries *))
\cvsubsection{((( i.title | escape_tex )))}
\begin{cventries}
((* for j in i.entries *))
\cventry
  {((( j.institution | escape_tex )))}
  {((( j.event | escape_tex )))}
  {((( j.location | escape_tex )))}
  {((( j.dates | escape_tex )))}
  {((( j.title | escape_tex | doi_to_url(j.doi, j.bibcode, "tex"))))}
((* endfor *))
\end{cventries}
((* endfor *))

%-------------------------------------------------------------------------------%
%	                                Honors and Awards                             %
%-------------------------------------------------------------------------------%
((* set sect = data.sections | select_by_attr_name("title", "Honors and Awards") *))
\cvsection{((( sect.title | escape_tex )))}
\begin{cvhonors}
((* for i in sect.entries *))
\cvhonor
  {((( i.name )))}
  {((( i.awarder )))}
  {}
  {((( i.dates )))}
((* endfor *))
\end{cvhonors}

%-------------------------------------------------------------------------------%
%	                                Software and Computing                        %
%-------------------------------------------------------------------------------%
((* set sect = data.sections | select_by_attr_name("title", "Software and Computing") *))
\cvsection{((( sect.title | escape_tex )))}
%--- Skills ---%
((* set sect1 = sect.entries | select_by_attr_name("title", "Skills") *))
\cvsubsection{((( sect1.title | escape_tex )))}
\begin{cvskills}
((* for i in sect1.entries *))
\cvskill{((( i.name | escape_tex )))}{((( i.value | escape_tex )))}
((* endfor *))
\end{cvskills}
%--- Open Source ---%
((* set sect2 = sect.entries | select_by_attr_name("title", "Open Source Contributions") *))
\cvsubsection{((( sect2.title | escape_tex )))}

A more complete record of my contributions is available on 
\href{https://github.com/((( data.preamble.contact.github )))}
     {GitHub}.
\begin{cventries}
((* for i in sect2.entries *))
\cventry
  {((( i.role | escape_tex )))}
  {((( i.name | escape_tex )))}
  {((( i.dates | escape_tex )))}
  {\href{https://github.com/((( i.repo )))}{github.com/((( i.repo )))}}
  {((( i.description )))}
((* endfor *))
\end{cventries}

%-------------------------------------------------------------------------------%
%	                                Teaching and Mentoring                        %
%-------------------------------------------------------------------------------%
\cvsection{Teaching and Mentoring}
%----------------- Students -----------------%
((* set sect = data.sections | select_by_attr_name("title", "Students Mentored") *))
\cvsubsection{((( sect.title | escape_tex )))}
\begin{cventries}
((* for i in sect.entries *))
\cventry
  {((( i.role | escape_tex )))}
  {((( i.name | escape_tex )))}
  {((( i.institution | escape_tex )))}
  {((( i.dates | escape_tex )))}
  {((( i.description | escape_tex )))}
((* endfor *))
\end{cventries}
%----------------- Courses -----------------%
((* set sect = data.sections | select_by_attr_name("title", "Teaching Experience") *))
\cvsubsection{((( sect.title | escape_tex )))}
\begin{cventries}
((* for i in sect.entries *))
\cventry
  {((( i.title | escape_tex )))}
  {((( i.class | escape_tex )))}
  {((( i.institution | escape_tex )))}
  {((( i.dates | escape_tex )))}
  {((( i.description | escape_tex )))}
((* endfor *))
\end{cventries}

%-------------------------------------------------------------------------------%
%	                                Memberships                                   %
%-------------------------------------------------------------------------------%
((* set sect = data.sections | select_by_attr_name("title", "Memberships") *))
\cvsection{((( sect.title | escape_tex )))}
\begin{cventries}
\begin{cvitems}
  ((* for i in sect.entries *))
  \item{((( i )))}
  ((* endfor *))
\end{cvitems}
\vspace{1em}
\end{cventries}

%-------------------------------------------------------------------------------
\end{document}
